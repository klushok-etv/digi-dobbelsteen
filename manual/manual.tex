\documentclass[12pt]{article}
\usepackage[english]{babel}
\usepackage{a4wide}             % Voor wat beter gevulde A4-tjes
%\usepackage[utf8]{inputenc}     
\usepackage{graphicx}           % To add pictures
\usepackage{wrapfig}
\usepackage[hmargin=3cm,vmargin=3.5cm]{geometry}
\usepackage{caption}            % For better captions
\usepackage{pdfpages}           % To add PDF pages
\usepackage{color}              % For colored text
\usepackage{subcaption}         % For subcaptions
\usepackage{amssymb}  
\usepackage{fancyhdr}
\usepackage{amsmath}
\usepackage[export]{adjustbox}
\usepackage{titling}
\usepackage{eurosym}
\usepackage{times}
\usepackage{textcomp}
\usepackage{colortbl}
\usepackage{eurosym}
% \usepackage{hyperref}
\usepackage{lastpage}
\usepackage{fancyhdr}
\usepackage[modulo]{lineno}
\usepackage[ampersand]{easylist}
\usepackage{soul}
\usepackage{marginnote}
\usepackage[normalem]{ulem}
\usepackage{multicol}
\usepackage{float}

\usepackage{siunitx}\sisetup{detect-all}

\usepackage{hyperref}
\hypersetup{
    colorlinks=true,
    linkcolor=blue,
    filecolor=blue,      
    urlcolor=blue,
    citecolor=blue,
    pdftitle={Christmas snowball}, % vast beter dan "overleaf example"
    pdfpagemode=FullScreen,
}


% Use XeLaTeX Compliler!!
\usepackage{fontspec}
\setmainfont{Arial}

\pagestyle{fancy}

\fancypagestyle{importedpages}{%
  \fancyhf{}% Clear header/footer
  \renewcommand{\headrulewidth}{0pt}% Remove header rule
  \renewcommand{\footrulewidth}{0pt}% Remove footer rule (default)
  \fancyfoot[C]{\raisebox{-3\baselineskip}[0pt][0pt]{\thepage}}% Lower page number into position
}

% \hypersetup{colorlinks,%
% 	citecolor=black,%
% 	filecolor=red,%
% 	linkcolor=black,%
% 	urlcolor=black}

%Header 
\setlength{\voffset}{-0.5in}% Afstand van de top tot de header (+1 inch)
\setlength{\headheight}{110pt}% Hoogte van de header
\setlength{\headsep}{10pt}% Afstand tussen header en document


\fancyhead{}
\fancyfoot{}
\fancyhead[RO,R]{\includegraphics[width=250pt]{ETVLogoRCMYK.pdf}}
\fancyfoot[CO,C]{page \thepage~of \pageref{LastPage}}
\renewcommand{\headrulewidth}{0pt}
\renewcommand{\footrulewidth}{0.4pt}



\begin{document}
\raggedright

\begin{center}
	\LARGE{Digi-dobbelstteen}\\
        \Large{Uitleg}\\
	\large{February 2024}\\
	% \normalsize{Tom Geerling}\\
	~\\
	\large{\emph{Klushok - Electrotechnische Vereeniging}}\\
\end{center}

\reversemarginpar

\begin{figure}[H]
	\centering
	\includegraphics[width=0.8\textwidth]{} % comment when printing in black and white
\end{figure}
\newpage

% \section*{For readers with experience and little time}



\section{Introduction}


\section{Features}




\section{Kit contents} \label{sec:KitContent}


\subsection{How to read SMD resistor markings}


\section{Assembly}


\subsection{Introduction to soldering}

\textit{This is a beginner's guide to SMD soldering. If you already have worked with SMD components, you can skip reading this section.}
\vspace{1ex}

SMD soldering may be different from soldering you may have done in the past. Everything related to placing one component is done on the same side of the PCB. There are no holes in the PCB that keep the component in place. You need to position the component and solder it in place at the same time. This may take some practice. The following steps help you get started.
\newline
The first step is to inspect the PCB. In case that dirt or oxide is present in the pads, you can clean the PCB with some Isopropyl Alcohol. The PCB for this project has gold plated pads, so corrosion will be minimal. If you need to clean, give the pads a light rub using some cloth or a cotton swab.
\newline
Decide which hand you want to use for positioning the components. You need to hold the soldering iron in your other hand. For two terminal components, orient the PCB in a way that the component is oriented left to right. Put a small amount of solder on the pad closest to the hand you are going to use for holding the soldering iron. To do this, heat up the pad using the soldering iron and feed a small amount of solder. The solder should form a small dome attached to the pad, not extending outside the pad.\newline
Now get the component. You can use tweezers to lift the plastic covering the component. Now pick up the component using tweezers. Use the soldering iron to melt the solder on the pad. At the same time, place the component on the pads. Remove the soldering iron, leaving the component in place.
\newline The component is now fixed in position. The last step is to solder the remaining terminal. Position the soldering iron at the unsoldered terminal and apply some solder. The solder should connect the component to the pad. Repeat for all terminals and components.

\subsection{Recommended order of assembly}




\clearpage 

\subsection{Component orientation}


\section{Verification}
After all components are in place, start by visually checking if all terminals are soldered. Fix any missed connections or loose components. \\
\vspace{1ex}
Next, slide the switch in the direction of P5. Now check for continuity between power and ground. 
The easiest way to check this is by using the continuity setting on your digital multimeter and measuring across the test points labeled Vcc and Gnd near the power connector.
If these lines are accidentally connected, this will short the power supply.

\vspace{1ex}
\textit{\textbf{Note:} When measuring continuity across a capacitor, a continuity will be measured for a short period of time. This is due to the charging of the capacitors and is expected (and thus correct) behaviour.}

% \begin{figure}[H]
%     \centering
%     \includegraphics[width=.5\textwidth]{images/final_assembly.png}
%     \caption{Final circuit (It should look like this)}
%     \label{fig:final_assembly}
% \end{figure}

%\newpage
%\section{Visual Inspection, checking and cleaning}
%\begin{itemize}
%	\item 
%\end{itemize}

%Now you're good to go. If you thought it was easy than try to solder the battery assembly.

%\textbf{if you dont want to break your hands soldering} than make a short connnection on \textbf{pad Jx}. This will bypass the battery assembly.
%If you later on want to try the battery assembly

\section{Enjoy}



\clearpage
\appendix


